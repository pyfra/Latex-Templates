% Tables
\documentclass{article}
\title{Tables}
\date{}
\usepackage{multirow}

\begin{document}
	\maketitle
	\section{Simple Tables}	
	\begin{center}
		\begin{tabular}{ | l | c | r |}
			\hline
			1 & 2 & 3 \\ \hline
			4 & 5 & 6 \\ \hline
			7 & 8 & 9 \\
			\hline
		\end{tabular}
	
	\begin{tabular}{|r|l|}
		\hline
		7C0 & hexadecimal \\
		3700 & octal \\ \cline{2-2}
		11111000000 & binary \\
		\hline \hline
		1984 & decimal \\
		\hline
	\end{tabular}
	\end{center}

	\subsection{Wrapping Text}
	In order to allow for wrapping text, we will needto specify the width of the column. 
	
	\begin{center}
		\begin{tabular}{ | l | l | l | p{5cm} |}
			\hline
			Day & Min Temp & Max Temp & Summary \\ \hline
			Monday & 11C & 22C & A clear day with lots of sunshine.  
			However, the strong breeze will bring down the temperatures. \\ \hline
			Tuesday & 9C & 19C & Cloudy with rain, across many northern regions. Clear spells 
			across most of Scotland and Northern Ireland, 
			but rain reaching the far northwest. \\ \hline
			Wednesday & 10C & 21C & Rain will still linger for the morning. 
			Conditions will improve by early afternoon and continue 
			throughout the evening. \\
			\hline
		\end{tabular}
	\end{center}
	
	\subsubsection{Manually specify where to break a line with parbox}
	\begin{tabular}{cc}
		NOTHING & \parbox[t]{5cm}{before wrapping\\after breaking\\ breaking again}`
	\end{tabular}

	\subsection{Changing space between rows}
	\begin{table}[t]
		\renewcommand{\arraystretch}{1.5}
		\begin{center}
		\begin{tabular}{ | l | c | r |}
			\hline
			1 & 2 & 3 \\ \hline
			4 & 5 & 6 \\ \hline
			7 & 8 & 9 \\
			\hline
		\end{tabular}
		\end{center}
	\caption{Table with space 1.5}
	\end{table}

		\begin{table}
		\renewcommand{\arraystretch}{5}
		\begin{center}
			\begin{tabular}{ | l | c | r |}
				\hline
				1 & 2 & 3 \\ \hline
				4 & 5 & 6 \\ \hline
				7 & 8 & 9 \\
				\hline
			\end{tabular}
		\end{center}
		\caption{Table with space 5}
	\end{table}
	

\section{Multi-columns}
\subsection{First Example}
\begin{center}
\begin{tabular}{cc|c|c|c|c|l}
	\cline{3-6}
	& & \multicolumn{4}{ c| }{Primes} \\ \cline{3-6}
	& & 2 & 3 & 5 & 7 \\ \cline{1-6}
	\multicolumn{1}{ |c  }{\multirow{2}{*}{Powers} } &
	\multicolumn{1}{ |c| }{504} & 3 & 2 & 0 & 1 &     \\ \cline{2-6}
	\multicolumn{1}{ |c  }{}                        &
	\multicolumn{1}{ |c| }{540} & 2 & 3 & 1 & 0 &     \\ \cline{1-6}
	\multicolumn{1}{ |c  }{\multirow{2}{*}{Powers} } &
	\multicolumn{1}{ |c| }{gcd} & 2 & 2 & 0 & 0 & min \\ \cline{2-6}
	\multicolumn{1}{ |c  }{}                        &
	\multicolumn{1}{ |c| }{lcm} & 3 & 3 & 1 & 1 & max \\ \cline{1-6}
\end{tabular}
\end{center}

\subsection{Second Example}
\begin{center}
\begin{tabular}{llr}
	\hline
	\multicolumn{2}{c}{Item} \\
	\cline{1-2}
	Animal    & Description & Price (\$) \\
	\hline
	Gnat      & per gram    & 13.65      \\
	& each        & 0.01       \\
	Gnu       & stuffed     & 92.50      \\
	Emu       & stuffed     & 33.33      \\
	Armadillo & frozen      & 8.99       \\
	\hline
\end{tabular}
\end{center}

\subsection{Third Example}
\begin{table}[p]
	\centering
	\begin{tabular}{|c|c|c|c|c|c|c|c|}
		\hline
		\multirow{4}{*}{Surface} & \multirow{4}{*}{Material} & \multicolumn{6}{c|}{Hardness}           \\
		\cline{3-8}
		&                    & \multicolumn{2}{c|}{Soft} & \multicolumn{2}{c|}{Medium} & \multicolumn{2}{c|}{Hard} \\
		\cline{3-8}
		&                    & \multicolumn{6}{c|}{Factors}                                               \\
		\cline{3-8}
		&                           &  A  &   B     &  A   &   B   &   A    &  B     \\
		\hline
		\multirow{2}{*}{Base Metal} &    Cast Iron    &  10 &   20    & 30   & 40    &  50    &  60      \\
		\cline{2-8}
		&  Steel           &  70 &   80    & 90   & 100   &  110   &  120     \\
		\hline
		\multirow{2}{*}{Dressing} &        Cast Iron  & 130 &  140    & 150  &  160  &  170   &  180     \\
		\cline{2-8}
		&          Steel            &  170&  160    & 150  &  140  &  130   &  120     \\
		\hline
		\multirow{2}{*}{S Dressing} &    Cast Iron    &  110&   100   & 90   & 80    &  70    &  60      \\
		\cline{2-8}
		&      Steel                &  50 &  50     & 30   & 20    & 10     &0\\
		\hline
	\end{tabular}
	\caption{Table of Factors for Use in Speed Formulas}
\end{table}


\end{document}